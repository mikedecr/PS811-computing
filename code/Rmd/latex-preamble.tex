% there are a few ways to include latex options.
% 1: as a preamble file (like this one)
% 2: directly into the header-includes YAML variable (global)
% 3: output templates. For more info on templates,
%    see <https://bookdown.org/yihui/rmarkdown/rticles-templates.html>
%    and <https://bookdown.org/yihui/rmarkdown/rticles-usage.html>

% These packages control the fonts used in PDF output.
% The standard LaTeX engine (PDFLaTeX) has a limited selection of fonts.
% See <https://www.tug.org/FontCatalogue/>

\usepackage{libertine}                 % serif font
\usepackage{libertinust1math}          % matching math font
\usepackage[scaled = 0.95, varqu]{zi4} % teletype (monospace/code) font

% The XeLaTeX engine gives you more control to specify fonts
%   at the expense of having poorer typesetting principles
%   (but xelatex typesetting is getting better)

% this package enables decimal-aligned table columns
\usepackage{dcolumn}

% the default PDF template for R Markdown loads a lot of convenient packages
% so you don't actually have to mess with them all that much.


